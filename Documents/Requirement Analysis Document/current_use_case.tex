

		\begin{tabular}{|l|p{13cm}|}
			\hline
			\textbf{Use case name}  & \textbf{UC 1 - Ricerca di code smell} \\ \hline
			\textbf{Participating actors}  & \textbf{Sviluppatore} \\ \hline
			\textbf{Entry condition}  & L'utente avvia con successo Intellij.\\ \hline 			
			\textbf{Flow of events}  & 
			\begin{tabular}{p{6cm}p{6cm}p{6cm}}
				\centering\textbf{UTENTE}  & \centering\textbf{SISTEMA} &\\ \\
				\textbf{1.}\hspace{0.3cm}L'utente su IntelliJ, tramite menù a tendina del plug-in, richiede l'analisi locale sulla classe o sul package designato che vuole correggere, evidenziando che tipologia di code smell analizzare.\\ 
				&\textbf{2.}\hspace{0.3cm}Il sistema analizza il codice ed effettua il calcolo delle metriche strutturali. Sulla base dei dati estrapolati vengono poi mostrati i valori riguardanti le metriche di qualità.\\
				\textbf{3.}\hspace{0.3cm}L'utente prende visione del risultato dell'analisi tramite una tabella e varie RadarMap riportanti i 5 termini più frequenti . La tabella riporta la posizione del code smell identificato e le metriche di qualità calcolate(LOC, WNC, RFC, CBO, LCOM, NOA, NOM, NOPA, NOP). L'utente può inoltre, tramite apposito pulsante "Perform Refactor", richiedere il refactoring sullo smell. \\	\\		
			\end{tabular}\\ \hline		
			\textbf{Exit condition}  & Lo sviluppatore visiona correttamente i dati e il codice affetto da smell.\\ \hline 			
			\textbf{Exception condition} & Nel passo 1, in caso di codice java non corretto dal punto di vista sintattico, semantico o lessicale si verifica lo UC 1.1.\\ \hline
			\textbf{Priority}  & Alta \\ \hline 	
			\textbf{Quality requirements}  & 
					\begin{itemize}
						\item Tempo di elaborazione massimo: 3 minuti.
						\item Il sistema è in grado di rilevare se il codice è java.
					\end{itemize}
							
			\\ \hline 
		\end{tabular}

		\vspace{1cm}

		\begin{tabular}{|l|p{13cm}|}
			\hline
			\textbf{Use case name}  & \textbf{UC 1.1 - Analisi codice non compilabile} \\ \hline
			\textbf{Participating actors}  & \textbf{Sviluppatore} \\ \hline
			\textbf{Entry condition}  & L'utente avvia l'analisi del codice.\\ \hline 			
			\textbf{Flow of events}  & 
			\begin{tabular}{p{6cm}p{6cm}p{6cm}}
				\centering\textbf{UTENTE}  & \centering\textbf{SISTEMA} &\\ \\
				\textbf{1.}\hspace{0.3cm}L'utente attende l'elaborazioni richieste dal sistema.\\ 
				&\textbf{2.}\hspace{0.3cm}Il sistema tenta di analizzare il codice ma, trovando testo non compilabile, annulla l'operazione e mostra un messaggio di errore.\\
				\textbf{3.}\hspace{0.3cm}L'utente prende visione del messaggio che indica il fallimento dell'operazione. \\	\\		
			\end{tabular}\\ \hline		
			\textbf{Exit condition}  & Il sistema termina l'esecuzione senza effettuare alcuna modifica\\ \hline 			
			\textbf{Exception condition} & \\ \hline
			\textbf{Priority}  & Alta \\ \hline 	
			\textbf{Quality requirements}  & 
			\begin{itemize}
				\item Tempo di elaborazione massimo: 3 minuti.
			\end{itemize}
			
			\\ \hline 
		\end{tabular}
	
	 	\newpage 
	 	
			%USE CASE 2.1 - "CORREZIONE CODE SMELL SOLO FE"
		
		\begin{tabular}{|l|p{13cm}|}
			\hline
			\textbf{Use case name}  & \textbf{UC 2.1 - Correzione Code smell - Feature Envy} \\ \hline
			\textbf{Participating actors}  & \textbf{Sviluppatore} \\ \hline
			\textbf{Entry condition}  & L’utente ha visualizzato il codice affetto da smell  \\  \hline
			\textbf{Flow of events}  &  
			\begin{tabular}{p{6cm}p{6cm}p{6cm}}
				\centering \textbf{UTENTE} & \centering \textbf{SISTEMA} & \\
				\textbf{1.}\hspace{0.3cm}L’utente, dopo aver preso visione dei risultati dati del caso UC1.0 sceglie i parametri su cui intervenire per lo smells di tipo “Feature Envy” e avvia gli automatismi per la correzione automatica premendo il tasto “Refactor”.
				\\ \\ &
				\textbf{2.}\hspace{0.3cm}Il sistema, mostra una schermata di riepilogo con delle sezioni mostranti una proposta di correzione   confrontata con la situazione precedente all’analisi, e un tasto per apportare tali modifiche. \\ \\
				
				\textbf{3.}\hspace{0.3cm}L’utente, può scegliere di accettare tali modifiche consigliate dal sistema cliccando il tasto "Applica modifiche".
				\\ \\ &
				\textbf{4.}\hspace{0.3cm}Il sistema, applica la soluzione che precedentemente ha mostrato all'utente. \\ \\
				
				\textbf{5.}\hspace{0.3cm}L'utente prende visione di tutte le modifiche effettuate dal sistema. \\ \\
				
			\end{tabular}\\ \hline
			
						\textbf{Exit condition}  & Il codice è stato correttamente modificato.\\ \hline 
			\textbf{Exception condition} &Se nel passo 1 o 3 l'utente chiude la finestra di dialogo e non procede alla correzione si verifica lo UC 2.3\\ \hline
			\textbf{Priority}  & Alta \\ \hline 	
			\textbf{Quality requirements}  & 
			\\ \hline 
		\end{tabular}
		
		%USE CASE 2.2 - "CORREZIONE CODE SMELL SOLO MC"
		
		\begin{tabular}{|l|p{13cm}|}
			\hline
			\textbf{Use case name}  & \textbf{UC 2.2 - Correzione Code smell - Misplaced Class} \\ \hline
			\textbf{Participating actors}  & \textbf{Sviluppatore} \\ \hline
			\textbf{Entry condition}  & L’utente ha visualizzato il codice affetto da smell  \\  \hline
			\textbf{Flow of events}  &  
			\begin{tabular}{p{6cm}p{6cm}p{6cm}}
				\centering \textbf{UTENTE} & \centering \textbf{SISTEMA} & \\
				\textbf{1.}\hspace{0.3cm}L’utente, dopo aver preso visione dei risultati dati del caso UC1.0 sceglie i parametri su cui intervenire per lo smells di tipo “Misplaced Class” e avvia gli automatismi per la correzione automatica premendo il tasto “Refactor”.
				\\ \\ &
				\textbf{2.}\hspace{0.3cm}Il sistema, mostra una schermata di riepilogo con delle sezioni mostranti una proposta di correzione   confrontata con la situazione precedente all’analisi, e un tasto per apportare tali modifiche. \\ \\
				
				\textbf{3.}\hspace{0.3cm}L’utente, può scegliere di accettare tali modifiche consigliate dal sistema cliccando il tasto "Applica modifiche".
				\\ \\ &
				\textbf{4.}\hspace{0.3cm}Il sistema, applica la soluzione che precedentemente ha mostrato all'utente. \\ \\
				
				\textbf{5.}\hspace{0.3cm}L'utente prende visione di tutte le modifiche effettuate dal sistema. \\ \\
				
			\end{tabular}\\ \hline
			
			\textbf{Exit condition}  & Il codice è stato correttamente modificato.\\ \hline 
			\textbf{Exception condition} &Se nel passo 1 o 3 l'utente chiude la finestra di dialogo e non procede alla correzione si verifica lo UC 2.3\\ \hline
			\textbf{Priority}  & Alta \\ \hline 	
			\textbf{Quality requirements}  & 
			\\ \hline 
		\end{tabular}			
		
		\vspace{1cm} 
		
		\begin{tabular}{|l|p{13cm}|}
			\hline
			\textbf{Use case name}  & \textbf{UC 2.3 - Correzione annullata} \\ \hline
			\textbf{Participating actors} & \textbf{Sviluppatore} \\ \hline
			\textbf{Entry Condition} & L'utente chiude le finestre di dialogo e non procede alla correzione. \\ \hline
			\textbf{Flow of events} & 
			\begin{tabular}{p{6cm}p{6cm}p{6cm}}
				\centering \textbf{UTENTE} & \centering \textbf{SISTEMA} & \\
				\textbf{1.}\hspace{0.3cm}Lo Sviluppatore decide di non correggere il code smell. & \\
				& \textbf{2.}\hspace{0.3cm}Il Sistema non effettua alcuna operazione e vengono chiuse le finestre di dialogo. 
			\end{tabular} \\ \hline 
			\textbf{Exit condition} & Il Sistema termina l'operazione senza effettuare modifiche al codice.\\ \hline 
			\textbf{Exception condition} & \textbf{} \\ \hline
			\textbf{Priority}  & Alta \\ \hline 	
			\textbf{Quality requirements} & \textbf{} \\ \hline
		\end{tabular}
		
%\\sorgente, tipo, metriche  \\ \\	