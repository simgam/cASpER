%%%%%%%%%%%%%%%%%%%%%%%%%%%%%%%%%%%%%%%%%
% Academic Title Page
% LaTeX Template
% Version 2.0 (17/7/17)
%
% This template was downloaded from:
% http://www.LaTeXTemplates.com
%
% Original author:
% WikiBooks (LaTeX - Title Creation) with modifications by:
% Vel (vel@latextemplates.com)
%
% License:
% CC BY-NC-SA 3.0 (http://creativecommons.org/licenses/by-nc-sa/3.0/)
% 
% Instructions for using this template:
% This title page is capable of being compiled as is. This is not useful for 
% including it in another document. To do this, you have two options: 
%
% 1) Copy/paste everything between \begin{document} and \end{document} 
% starting at \begin{titlepage} and paste this into another LaTeX file where you 
% want your title page.
% OR
% 2) Remove everything outside the \begin{titlepage} and \end{titlepage}, rename
% this file and move it to the same directory as the LaTeX file you wish to add it to. 
% Then add \input{./<new filename>.tex} to your LaTeX file where you want your
% title page.
%
%%%%%%%%%%%%%%%%%%%%%%%%%%%%%%%%%%%%%%%%%

%----------------------------------------------------------------------------------------
%	PACKAGES AND OTHER DOCUMENT CONFIGURATIONS
%----------------------------------------------------------------------------------------

\documentclass[11pt]{article}

\usepackage[italian]{babel}

\usepackage[utf8]{inputenc} % Required for inputting international characters
\usepackage[T1]{fontenc} % Output font encoding for international characters

\usepackage{mathpazo} % Palatino font
\usepackage{graphicx}
\usepackage{fancyhdr}
\usepackage{tabularx}
\usepackage{geometry}

\geometry{legalpaper, margin=2.5cm}

\newcommand{\doctitle}{Change Request}
\newcommand{\docversion}{1.0}

\begin{document}
	
	%----------------------------------------------------------------------------------------
	%	TITLE PAGE
	%----------------------------------------------------------------------------------------
	
	\begin{titlepage} % Suppresses displaying the page number on the title page and the subsequent page counts as page 1
		\newcommand{\HRule}{\rule{\linewidth}{0.5mm}} % Defines a new command for horizontal lines, change thickness here
		
		\center % Centre everything on the page
		
		%------------------------------------------------
		%	Headings
		%------------------------------------------------
		
		\textsc{\LARGE Università degli Studi di Salerno}\\
		\textsc{\large Corso di Ingegneria del Software}\\[1.5cm] % Main heading such as the name of your university/college
		
		%------------------------------------------------
		%	Title
		%------------------------------------------------
		
		\HRule\\[0.4cm]
		
		{\huge\bfseries ASCETIC}\\ % Title of your document
		\vspace{0.2cm}
		{\large\bfseries Automated Code Smell Identification and Correction}\\[0.2cm] % Title of your document
		
		\HRule\\[1.5cm]
		
		\textsc{\Large \doctitle}\\[0.3cm] % Major heading such as course name
		
		%\textsc{\large Version \docversion}\\[0.5cm] % Minor heading such as course title
		
		
		%------------------------------------------------
		%	Logo
		%------------------------------------------------
		
		\vfill\vfill
		
		\includegraphics[width=0.5\textwidth]{../logo_temp.jpg}\\[1cm] % Include a department/university logo - this will require the graphicx package
		
		%------------------------------------------------
		%	Date
		%------------------------------------------------
		
		\vfill\vfill\vfill % Position the date 3/4 down the remaining page
		
		{\large\today} % Date, change the \today to a set date if you want to be precise
		
	
		
		%----------------------------------------------------------------------------------------
		
		\vfill % Push the date up 1/4 of the remaining page
		
	\end{titlepage}
	
	%----------------------------------------------------------------------------------------
	
	\pagestyle{fancy}
	\rhead{ASCETIC}
	%\lhead{\doctitle~v.~\docversion}
	\renewcommand{\headrulewidth}{0pt}
	
	
	\textbf{Coordinatore Progetto:}
	\begin{table}[h]
		\centering
		\begin{tabularx}{0.9\textwidth}{|X|X|}
			\hline
			\textbf{Nome}     & \textbf{Matricola} \\ \hline
			Manuel De Stefano &  0522500633\\ \hline
		\end{tabularx}
	\end{table}

	\vspace{0.5cm}
	
	\textbf{Partecipanti:}
	\begin{table}[h]
		\centering
		\begin{tabularx}{0.9\textwidth}{|X|X|}
			\hline
			\textbf{Nome}     & \textbf{Matricola} \\ \hline
			Amoriello Nicola &  0512104742\\ \hline
			Di Dario Dario &  0512104758\\ \hline
			Gambardella Michele Simone &  0512104502\\ \hline
			Iovane Francesco &  0512104550\\ \hline
			Pascucci Domenico &  0512102950\\ \hline
			Patierno Sara &  0512103460\\ \hline
		\end{tabularx}
	\end{table}

	\textbf{Revision History:}
	\begin{table}[h]
		\centering
		\begin{tabularx}{0.9\textwidth}{|p{2cm}|l|X|p{3cm}|}
			\hline
			\textbf{Data} & \textbf{Versione} & \textbf{Descrizione} & \textbf{Autore} \\ \hline
			10/10/2018 & 1.0  & Prima Stesura & Manuel De Stefano \\ \hline
		\end{tabularx}
	\end{table}


	\textbf{Change Request:}
	\begin{table}[h!]
		\centering
		\begin{tabularx}{0.9\textwidth}{|p{3cm}|p{2cm}|X|l|}
			\hline
			\textbf{Numero Change Request} & \textbf{Data Richiesta} & \textbf{Richiesta da} & \textbf{Priorità} \\ \hline
			CR-01 & 10/10/2018 & Manuel De Stefano & \textbf{Alta} \\ \hline
		\end{tabularx}
	\end{table}


	
	\section{Descrizione}
		Si richiede una manutenzione evolutiva di Ascetic (ex TACOR), un plugin per Intellij IDEA per individuazione e correzione
		automatica di code smell.

	\section{Scopo}
		Lo scopo di questa evoluzione consiste nell' aggiungere due nuove funzionalità al plugin, ovvero, la correzione automatica del code smell Blob e la correzione automatica del code smell Promiscuous Package.		
	
	\section{Conseguenze se non accettata}
		Il comportamento del plugin rimarrà invariato. Tuttavia, allo stato attuale, l'unico modo per correggere questo tipo di code smell è attraverso un refactoring manuale, che potrebbe essere eseguito in modo erroneo. Il plugin, infatti, per Blob e Promiscuous package fornisce solamente la funzionalità di individuazione, senza nemmeno proporre un potenziale refactoring.
	
\end{document}
