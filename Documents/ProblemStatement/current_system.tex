
	
	\subsection{Scenari}
	\begin{tabular}{|l|p{8cm}|p{1cm}|p{1.1cm}}
		\hline
		\textbf{Nome scenario}  & \textbf{SC 1: Ricerca Blob} \\ \hline
		\textbf{Istanze di attori partecipanti}  & \textbf{Manuel: Sviluppatore} \\ \hline
		\textbf{Flusso di eventi} & \begin{enumerate}
			\item Manuel, sviluppatore del codice vuole ricercare all’interno del codice Java potenziali code smells di tipo "Blob".
			\item All’interno di intelliJ, nella sezione dedicata ai packages, con il tasto destro Manuel seleziona tra i plug-ings “Ascetic”, e nella sottovoce sceglie “Blob”.
			\item Dopo la selezione della voce e un elaborazione, viene visualizzata a schermo una finestra contenente una radar map con i 5 termini maggiormente presenti e a più alta probabilità di smell, oltre ad una tabella riepilogativa.
			\item Siccome per questa tipologia di smell non c'è meccanismo di auto-fix, Manuel inizia ad intervenire sul codice in maniera manuale.
		\end{enumerate}\\ \hline
	\end{tabular}

	\vspace{1cm}
		
		
	\noindent\begin{tabular}{|l|p{8cm}|p{1cm}|p{1.1cm}|}
		\hline
		\textbf{Nome scenario}  & \textbf{SC 2: Ricerca Promiscuous Package} \\ \hline
		\textbf{Istanze di attori partecipanti}  & \textbf{Manuel: Sviluppatore} \\ \hline
		\textbf{Flusso di eventi}  & \begin{enumerate}
	 \item Manuel, sviluppatore del codice vuole ricercare all’interno del codice Java potenziali code smells di tipo "Promiscuous Package".
	 \item  All’interno di intelliJ, nella sezione dedicata ai packages, con il tasto destro Manuel seleziona tra i plug-ings “Ascetic”, e nella sottovoce sceglie “Promiscuous Package”. 
		\item Dopo la selezione della voce e un elaborazione, viene visualizzata a schermo una finestra contenente una radar map con i 5 termini maggiormente presenti e a più alta probabilità di smell, oltre ad una tabella riepilogativa.
		\item Fatto tesoro del risultato mostrato, Manuel inizia il fixing manuale del problema, data l'assenza di automatismi di fixing.
		\end{enumerate} \\ \hline
	\end{tabular}

	\vspace{0.5cm}
		
	\begin{tabular}{|l|p{8cm}|p{1cm}|p{1.1cm}|}
		\hline
		\textbf{Nome scenario}  & \textbf{SC 3: Ricerca Misplaced Class} \\ \hline
		\textbf{Istanze di attori partecipanti}  & \textbf{Manuel: Sviluppatore} \\ \hline
		\textbf{Flusso di eventi}  & \begin{enumerate}
		\item Manuel, sviluppatore del codice vuole ricercare all’interno del codice Java potenziali code smells di tipo Misplaced Class. 
		\item All’interno di intelliJ, nella sezione dedicata ai packages, con il tasto destro Manuel seleziona tra i plug-ings “Ascetic”, e nella sottovoce sceglie “Misplaced Class”. 
		\item Dopo la selezione della voce e un elaborazione, viene visualizzata a schermo una finestra,in cui si possono notare tre radar map che mostrano i topic del package attuale (in alto a sinistra), del package invidiato (in alto a destra) e del potenziale candidato a Misplaced Class (in basso a destra); nella parte bassa della schermata è presente anche una tabella contenente tutti i potenziali candidati individuati. Manuel seleziona la voce "Perform Refactoring" e una nuova finestra viene mostrata all’utente.
		\item  La nuova finestra avrà e due nuove radar map (per un totale di 5),che rappresentano una previsione rispettivamente dei topic implementati dal package attuale post-refactoring e i topic del package invidiato dopo il refactoring. Selezionando il bottone "Refactor Code" verrà applicata l’operazione suggerita alla classe affetta dallo smell che viene spostata nel package invidiato.
		\item La schermata si chiude con un avviso che il fix è stato applicato.
		\end{enumerate} \\ \hline
	\end{tabular}
	\vspace{0.5cm}
		
		\begin{tabular}{|l|p{8cm}|p{1cm}|p{1.1cm}|}
			\hline
			\textbf{Nome scenario}  & \textbf{SC 4: Ricerca FeaturesEnvy} \\ \hline
			\textbf{Istanze di attori partecipanti}  & \textbf{Manuel: Sviluppatore} \\ \hline
			\textbf{Flusso di eventi}  & \begin{enumerate}
				\item Manuel, sviluppatore del codice vuole ricercare all’interno del codice Java potenziali code smells di tipo Feature Envy. 
				\item All’interno di intelliJ, nella sezione dedicata ai packages, con il tasto destro Manuel seleziona tra i plug-ings “Ascetic”, e nella sottovoce sceglie “Feature Envy”.
				\item Dopo la selezione della voce e un elaborazione, viene visualizzata a schermo una finestra, si possono notare le tre radar map che mostrano i topic della classe attuale, della classe invidiata, e del potenziale candidato a Feature Envy. In più sarà presente anche una tabella contenente tutti i potenziali candidati individuati, selezionando la voce Perform Refactoring, una nuova finestra viene mostrata all’utente.
				\item  La nuova finestra avrà e due nuove radar map,che rappresentano rispettivamente i topic implementati dalla classe attuale dopo il refactoring e i topic della classe invidiata dopo il refactoring, selezionando il bottone Refactor Code verrà, quindi, applicata l’operazione suggerita, e quindi, il metodo affetto dallo smell sarà spostato nella classe invidiata.
				\item La schermata si chiude con un avviso che il fix è stato applicato.
				\item Manuel clicca su Refactor e dopo una breve elaborazione la finestra del plug-in si chiude e lo smell è stato fixato.\end{enumerate} \\ \hline
		\end{tabular}

	\newpage
	
		\subsection{Requisiti Funzionali}
		\paragraph{RF 1 - Ricerca code smell} 
		\begin{quote}Il sistema ricerca i code smell all'interno del codice java.	\begin{description}
				\item [RF 1.1:]Il sistema permette l'analisi di code smell di tipo Blob,
				\item [RF 1.2:]Il sistema permette l'analisi di code smell di tipo Misplaced Class
				\item [RF 1.3:]Il sistema permette l'analisi di code smell di tipo Feature Envy
				\item [RF 1.4:]Il sistema permette l'analisi di code smell di tipo Promiscuos Package
		\end{description}\end{quote}
	
		
		
		\paragraph{RF 2 - Correzione} 
		\begin{quote}Il sistema mostra una possibile soluzione al code smell.\begin{description}
				\item [RF 2.1:] Il sistema permette di corregere code smell di tipo Feature Envy.
				\item [RF 2.2:] Il sistema permette di corregere code smell di tipo Misplaced Class.
		\end{description}\end{quote}
		
		
		\paragraph{RF 3 - Refactoring} 
		\begin{quote}Il sistema permette allo sviluppatore di risolvere il code smell con degli automatismi del software.\begin{description}
				\item [RF 3.1:] Il sistema permette di risolvere code smell di tipo Feature Envy.
				\item [RF 3.2:] Il sistema permette di risolvere code smell di tipo Misplaced Class.
		\end{description}\end{quote}
		
	
			\paragraph{RF 4 - Metriche di Qualità} 
		\begin{quote}Il sistema permette di calcolare metriche di qualità.\begin{description}
				\item [RF 4.1:] Il sistema permette di calcolare metriche di qualità per i metodi.
				\item [RF 4.2:] Il sistema permette di calcolare metriche di qualità per le classi.
				\item [RF 4.3:] Il sistema permette di calcolare metriche di qualità per i package.
		\end{description}\end{quote}
	
			\paragraph{RF 5 - Estrazione dei Topic} 
		\begin{quote}Il sistema consente di estrarre i topic implementati.\end{quote}
			

		\subsection{Requisiti Non Funzionali}

\begin{list}{-}{}
	
	\item \textbf{RNF 1-Performance}\newline 
	Il sistema è concepito come utilizzabile da singolo utente, con un'interazione diretta, il tempo di risposta varia in base all'analisi proposta.
	 \newline 
	\item \textbf{RNF 2-Robustezza}
	\newline  Il sistema allo stato attuale non soddisfa il requisito di robustezza.
	\item \textbf{RNF 3-Sicurezza}
	\newline Il sistema si presenta sufficientemente sicuro in quanto il suo ambiente d'azione è locale, quindi non vi è la necessità di protezione da minacce esterne.
	\item \textbf{RNF 4-Usabilità}\newline 
	Il sistema presenta bassi criteri di usabilità, data una scarna e poco intuitiva GUI.
	\newline 
	\item \textbf{RNF 5-Manutenibilità}\newline 
	Il sistema non presenta alcun tipo di documentazione pregressa, oltre ad adottare scelte implementative poco adatte a tale scopo.
	\newline  
	\item \textbf{RNF 6-Implementazione}
	\newline Il sistema è realizzato interamente in linguaggio Java, sia parte back-end che front-end. 
\end{list}
	
