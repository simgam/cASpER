	
\subsection{Scenari}
\begin{quote}
	I seguenti scenari, appartenenti al Current Envirorment, non sono stati modificati:\\
	SC3 e SC4.	\\ \\	
	Gli scenari SC1 e SC2 sono modificati in seguito alle nuove funzionalità apportate al sistema:
\end{quote}

 \begin{tabular}{|l|p{8cm}|p{1cm}|p{1.1cm}|}
	\hline
	\textbf{Nome scenario}  & \textbf{SC1 - Ricerca BLOB} \\ \hline
	\textbf{Istanze di attori partecipanti}  & \textbf{Manuel: Sviluppatore} \\ \hline
	\textbf{Flusso di eventi}  & \begin{enumerate}
		\item  Manuel, sviluppatore del codice, vuole ricercare all’interno del codice Java potenziali code smells di tipo BLOB. 
		\item  All’interno di IntelliJ, nella sezione dedicata ai packages, con il tasto destro Manuel seleziona tra i plug-ings “Ascetic”, e nella sottovoce sceglie “Blob”. 
		\item Dopo la selezione della voce e un'elaborazione, viene visualizzata a schermo una finestra con una Radar-Map grafica che mostra i 5 termini più frequenti della classe attuale e una tabella che elenca i potenziali BLOB individuati all'interno del package. 
		\item Manuel clicca su "Check Solution" e viene aperta una nuova schermata con 3 Radar-Maps, una identica a quella della precedente schermata, e due che mostrano i 5 termini più frequenti delle due classi ottenute dalla possibile soluzione applicata.
		\item Manuel clicca su "Fix Code" per applicare la soluzione che il sistema ha elaborato ed eseguire un refactoring sul codice. 
		\item La schermata precedente è chiusa e il codice viene corretto. Una finestra di notifica avvisa dell'avvenuta correzione.
		\end{enumerate} \\ \hline
\end{tabular}
\newpage

\begin{tabular}{|l|p{8cm}|p{1cm}|p{1.1cm}|}
	\hline
	\textbf{Nome scenario}  & \textbf{SC2 - Ricerca Promiscuous Package} \\ \hline
	\textbf{Istanze di attori partecipanti}  & \textbf{Manuel: Sviluppatore} \\ \hline
	\textbf{Flusso di eventi}  & 
		\begin{enumerate}
		\item Manuel, sviluppatore del codice, vuole ricercare all’interno del codice Java potenziali code smells di tipo Promiscuous Package.
		
		\item All’interno di IntelliJ, nella sezione dedicata ai packages, con il tasto destro Manuel seleziona tra i plug-ings “Ascetic”, e nella sottovoce sceglie “Promiscuous Package”. 
		
		\item Dopo la selezione della voce e un'elaborazione, viene visualizzata a schermo una finestra con una Radar-Map grafica che mostra i 5 termini più presenti del package attuale e una tabella che elenca i potenziali Promiscuous Package individuati. 
		
		\item Manuel clicca su "Check Solution" e viene aperta una nuova schermata con 3 RadarMaps dove una mostra sempre i 5 termini più presenti del package attuale, in più vengono mostrate altre 2 Radar-Maps che mostrano i 5 termini più frequenti dei due nuovi package ottenuti dalla possibile soluzione applicata.
		
		\item Manuel clicca su "Fix Code" per applicare la soluzione che il sistema ha elaborato ed eseguire un refactoring sul codice.
		
		\item La schermata precedente è chiusa e il codice viene corretto. Una finestra di notifica avvisa dell'avvenuta correzione.
		
		\end{enumerate}\\ \hline
\end{tabular}

\newpage
		\subsection{Requisiti Funzionali}
		
		\begin{list}{-}{}
			\item{I seguenti requisiti funzionali, appartenenti al Current Envirorment, non sono stati modificati:}	
		\end{list}
		\begin{quote}	
			\begin{description}		
				\item[RF 1:]				
				\begin{list}{}{}
					\begin{quote}
						\item{RF 1.1}
					\end{quote}					
				\end{list}
			
				\item[RF 2:]				
				\begin{list}{}{}
					\begin{quote}
						\item{RF 2.1}
						\item{RF 2.2}		
					\end{quote}					
				\end{list}
			
				\item[RF 3:]				
				\begin{list}{}{}
					\begin{quote}
						\item{RF 3.1}
						\item{RF 3.2}		
					\end{quote}	
				\end{list}
				\item[RF 4:]	
				\begin{list}{}{}
						\begin{quote}
						\item{RF 4.1}
						\item{RF 4.2}		
					\end{quote}	
				\end{list}
				\item[RF 5]	
				\begin{list}{}{}
				\end{list}
			\end{description}
		\end{quote}
	
    	\begin{list}{-}{}
    		\item \textbf{RF 2 - Correzione} 
    		\begin{quote}Il sistema consente la creazione di una possibile soluzione ai quattro principalei tipi di code smell.\end{quote}
    		\begin{quote}
    			\begin{description}    		 	
    				\item [RF 2.3:]Il sistema fornisce una possibile soluzione al code smell di tipo Blob. 
    				\item[RF 2.4:] Il sistema fornisce una possibile soluzione al code smell di tipo Promiscuous Package.     			
    			\end{description}
    		\end{quote}
    		\item \textbf{RF 3 - Refactoring:} 
    		\begin{quote}Il sistema consente la creazione di una possibile soluzione ai quattro principali tipi di code smell.\end{quote}
    		\begin{quote}
    			\begin{description}
	    			\item[RF 3.3:]Il sistema permette allo sviluppatore di risolvere il code smell di tipo Blob.
	    			\item[RF 3.4:]Il sistema permette allo sviluppatore di risolvere il code smell di tipo Promiscuous Package.	
    			\end{description}
    		\end{quote}
    	\end{list}	
    
    \newpage	
    
		\subsection{Requisiti Non Funzionali}
		\begin{list}{-}{}
		
			\item \textbf{RNF 1 - Performance}
			\newline Il sistema deve garantire brevi tempi di risposta, in particolare nelle operazioni di analisi del codice.  
			\item \textbf{RNF 2 - Robustezza}
			\newline Il sistema è in grado di funzionare correttamente anche in situazioni anomale o in caso di uso scorretto, notificando l'utente della situazione erronea rilevata, ma senza terminare la propria esecuzione. 
			\item \textbf{RNF 3 - Sicurezza}
			\newline Il sistema si presenta sufficientemente sicuro in quanto il suo ambiente d'azione è locale, quindi non vi è la necessità di protezione da minacce esterne.
			\item \textbf{RNF 4 - Usabilità}
			\newline Il sistema risulta essere di facile utilizzo. L'interfaccia grafica risulta essere intuitiva, agevolando il lavoro dello sviluppatore. 
			\item \textbf{RNF 5 - Manutenibilità}
			\newline Il sistema presenta un elevato grado di manutenibilità, in particolare, favorisce l'aggiunta di nuove funzionalità.
			\item \textbf{RNF 6 - Implementazione}
			\newline Il sistema è realizzato interamente in linguaggio Java, sia parte back-end che front-end. 
	\end{list}

