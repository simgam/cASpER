Durante il ciclo di vita di un software, le modifiche costituiscono una parte essenziale ed inevitabile. La manutenzione, volta alla correzione di bug o all'integrazione di nuove funzionalità, porta ad un graduale deperimento del codice, il quale non inficia il corretto funzionamento del software bensì introduce delle debolezze di progettazione: i cosiddetti Code Smell. Per questo motivo nasce ASCETIC, un plug-in sviluppato per l'IDE intelliJ IDEA che consente di analizzare il progetto, rilevando 4 possibili tipologie di smell(Blob, Promiscuous Package, Feature Envy, Misplaced Class), ed effettuare un'eventuale correzione automatica. \\ ASCETIC (inizialmente TACOR) è un'opera di reingegnerizzazione, ed offre allo sviluppatore la possibilità di: 
\begin{itemize}
	\item Identificare le quattro sopracitate tipologie di Code Smell.
	\item Correggere automaticamente i Code Smell trovati dal sistema. 
	\item Far ignorare al sistema determinati Code Smell, selezionati ad hoc dallo sviluppatore, trovati in fase d'analisi. 
	\item Porre nella sezione "To do" di IntelliJ, tramite la funzione Reminder, gli Smell che lo sviluppatore decide di correggere manualmente in un secondo momento. 
\end{itemize}
Allo stato attuale ASCETIC usufruisce di una tecnica di analisi testuale per il rilevamento dei vari Code Smell: il primo step consiste nell'estrazione del contenuto testuale che caratterizza le componenti del codice, facendo una cernita degli elementi necessari per l'analisi, ossia i commenti e gli identificatori. Questi ultimi vengono normalizzati, ed infine le parole normalizzate vengono pesate in base allo schema tf-idf. Le componenti normalizzate vengono analizzate dallo Smell Detector, basato su LSI, che trasforma le componenti del codice in vettori di termini presenti in un dato software.  Tali vettori vengono proiettati in un K-spazio , ridotto appositamente per limitare l'effetto del rumore testuale. La dimensione di tale spazio viene determinata usando l'euristica di Kuhn. Infine la somiglianza fra i documenti viene calcolata come il coseno dell'angolo tra i due vettori. I valori vengono calcolati in maniera differente a seconda dello Smell da individuare per ottenere le probabilità che la porzione di codice sia affetta da tale Smell. Tali probabilità vengono convertite in valori booleani per indicare se un componente sia affetto o meno da Code Smell e richieda quindi una correzione. La scelta di questo metodo è dettata sia dalla validità comprovata di tale tecnica, sia dalla mancanza momentanea di alternative valide per effettuare il procedimento svolto dal plug-in in fase d'analisi. Uno degli obiettivi di sviluppo però è rendere ASCETIC flessibile alle modifiche ed all'introduzione di eventuali nuove tecniche per la gestione dell'analisi del codice. Per quanto riguarda il procedimento di refactoring, ASCETIC utilizza le API di IntelliJ per la manipolazione del codice sorgente. 
